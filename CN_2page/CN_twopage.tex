%-------------------------
% Resume in Latex
% Author : Sourabh Bajaj
% License : MIT
% Edited : lzhbrian
%------------------------

\documentclass[letterpaper,11pt]{article}

\usepackage{CJK}

\usepackage{latexsym}
\usepackage[empty]{fullpage}
\usepackage{titlesec}
\usepackage{marvosym}
\usepackage[usenames,dvipsnames]{color}
\usepackage{verbatim}
\usepackage{enumitem}
\usepackage[pdftex]{hyperref}
\usepackage{fancyhdr}


\pagestyle{fancy}
\fancyhf{} % clear all header and footer fields
\fancyfoot{}
\renewcommand{\headrulewidth}{0pt}
\renewcommand{\footrulewidth}{0pt}

% Adjust margins
\addtolength{\oddsidemargin}{-0.5in}
\addtolength{\evensidemargin}{-0.375in}
\addtolength{\textwidth}{1in}
\addtolength{\topmargin}{-.5in}
\addtolength{\textheight}{1.0in}

\urlstyle{same}

\raggedbottom
\raggedright
\setlength{\tabcolsep}{0in}

% Sections formatting
\titleformat{\section}{
  \vspace{-4pt}\scshape\raggedright\large
}{}{0em}{}[\color{black}\titlerule \vspace{-5pt}]

%-------------------------
% Custom commands
\newcommand{\resumeItem}[2]{
  \item\small{
    \textbf{#1}{ #2 \vspace{-2pt}}
  }
}

\newcommand{\resumeSubheading}[4]{
  \vspace{-1pt}\item
    \begin{tabular*}{0.97\textwidth}{l@{\extracolsep{\fill}}r}
      \textbf{#1} & #2 \\
      \textit{\small#3} & \textit{\small #4} \\
    \end{tabular*}\vspace{-5pt}
}

%%%
% add by lzhbrian
\newcommand{\resumeSubheadingProject}[2]{
  \vspace{-1pt}\item
    \begin{tabular*}{0.97\textwidth}{l@{\extracolsep{\fill}}r}
      \textbf{#1} & \emph{\small#2} \\
    \end{tabular*}\vspace{-5pt}
}


\newcommand{\resumeSubheadingedu}[7]{
  \vspace{-1pt}\item
    \begin{tabular*}{0.97\textwidth}{l@{\extracolsep{\fill}}r}
      \textbf{#1} & #2 \\
      ~~{\small#3} {\small#4} & {\small #5} \\
      ~~{\small#6} {\small#7} &\\
    \end{tabular*}\vspace{-5pt}
}
\newcommand{\resumeSubheadinghonor}[3]{
  \vspace{-1pt}\item
    \begin{tabular*}{0.97\textwidth}{l@{\extracolsep{\fill}}r}
      \textbf{#1} {#2}\\
      ~~\textit{\small#3}\\
    \end{tabular*}\vspace{-5pt}
}
\newcommand{\resumeSubheadinghonortwo}[4]{
  \vspace{-1pt}\item
    \begin{tabular*}{0.97\textwidth}{l@{\extracolsep{\fill}}r}
      \textbf{#1} {#2}\\
      ~~\textit{\small#3}\\
      ~~\textit{\small#4}\\
    \end{tabular*}\vspace{-5pt}
}
\newcommand{\sectionwithbox}[1]{
  \section[]{\colorbox[rgb]{0.8,0.8,0.8}{\textbf{#1}}}
}
\newcommand{\resumeSubheadinghonorone}[2]{
  \vspace{-1pt}\item
    \begin{tabular*}{0.97\textwidth}{l@{\extracolsep{\fill}}r}
      \textbf{#1} {#2}\\
    \end{tabular*}\vspace{-5pt}
}
\newcommand{\resumeSubheadingPublication}[1]{
  \vspace{-1pt}\item
      {#1}
    \vspace{-5pt}
}

%%%

\newcommand{\resumeSubItem}[2]{\resumeItem{#1}{#2}\vspace{-4pt}}

\renewcommand{\labelitemii}{$\circ$}

\newcommand{\resumeSubHeadingListStart}{\begin{itemize}[leftmargin=*]}
\newcommand{\resumeSubHeadingListEnd}{\end{itemize}}
\newcommand{\resumeItemListStart}{\begin{itemize}}
\newcommand{\resumeItemListEnd}{\end{itemize}\vspace{-5pt}}

%-------------------------------------------
%%%%%%  CV STARTS HERE  %%%%%%%%%%%%%%%%%%%%%%%%%%%%
%\AtBeginDocument{\csname CJK*\endcsname{UTF8}{gbsn}} 
%\AtEndDocument{\clearpage\csname endCJK*\endcsname}


\begin{document}
\begin{CJK*}{UTF8}{gbsn}

%----------HEADING-----------------
\begin{tabular*}{\textwidth}{l@{\extracolsep{\fill}}r}
  \color{blue}\textbf{\href{http://lzhbrian.me/}{\Large 林子恒~Tzu-Heng Lin (Brian) \small Maker, Hacker}} & 
  个人网站 : \color{blue}\href{http://lzhbrian.me/}{http://lzhbrian.me} \\
  \small~~~~出生于 \textbf{台湾}, 长大于 \textbf{上海}, 目前在 \textbf{北京}学习 & 邮箱 : \href{mailto:lzhbrian@gmail.com}{lzhbrian@gmail.com}\\
  
   ~{目前} : 清华大学电子工程系~本科三年级~(预计~2018 年毕业) & 手机 : +86-131-2192-9165 \\

   ~{求职} : 2017 年暑期实习~(软件工程/研发) &
   	\color{blue}\href{http://www.linkedin.com/in/lzhbrian}{\textbf{[Linkedin领英]}}
	\color{blue}\href{http://www.github.com/lzhbrian}{\textbf{[Github]}}
\end{tabular*}

%-----------EDUCATION-----------------
%\section{\colorbox[rgb]{0.8,0.8,0.8}{\makebox(70,10){Education}}}
  \sectionwithbox{教育经历}
  \resumeSubHeadingListStart
    \resumeSubheadingedu
      {清华大学}{中国~北京}
      {工学学士~-~电子工程}{~GPA: 84/100}{2014.8 - 2018.6}
      {管理学学士~-~工商管理 (第二学位)}{~GPA: 87/100}
  \resumeSubHeadingListEnd

%--------PROGRAMMING SKILLS------------
\sectionwithbox{技能}
  \resumeSubHeadingListStart
    \item{
      \textbf{编程语言}{: Python, C/C++, Matlab, R}
      \hfill
      \textbf{工具}{: Git, Hadoop, Weka, MySQL, Tableau}
    }
  \resumeSubHeadingListEnd



%-----------EXPERIENCE-----------------
\sectionwithbox{经验~与~项目}
  \resumeSubHeadingListStart
  
    \resumeSubheading
      {清华大学~\color{blue}\href{http://fi.ee.tsinghua.edu.cn}{[实验室网站]}}{中国~北京}
      {\textbf{研究助理} @ 未来通信与网络实验室}{2016.4 - Present (9 mo)}
      \resumeItemListStart

        \resumeItem{网络视频服务用户行为大数据挖掘}{(已完成, 论文已提交)}\\
        \vspace{1mm}
          {- 数据挖掘:一个\underline{TB级别的数据集,包括~2~个月,1~千万用户}在各种视频网站的观看记录,包括优酷,爱奇艺,搜狐,乐视,腾讯,看看,等等}\\
          {- 研究用户切换视频网站观看的行为,发现潜在迁移的原因}\\
          {- 负责绝大多数的数据处理以及分析工作} \\
          {~~~~- 使用~Java~编写 \underline{上百个 Hadoop MapReduce 程序}, 以及使用~Python~处理资料}\\
          {~~~~- 使用~Matlab~,~R~分析及可视化数据}\\

        \resumeItem{在线购物用户行为大数据挖掘}{(已完成, 论文已提交)}\\
        \vspace{1mm}
          {- 数据挖掘:一个包括\underline{~30~天,100万京东用户}的浏览商品记录}\\
          {~~~~- 发现用户浏览商品时的潜在行为} \\
          {~~~~- 负责绝大多数的数据处理以及分析工作~(与前一个项目技能点相同)}\\
          
      \resumeItemListEnd
  
    \resumeSubheading
      {中央研究院~\color{blue}\href{http://mmnet.iis.sinica.edu.tw}{[实验室网站]}}{台湾~台北}
      {\textbf{暑期实习生} @ 资料洞察实验室}{2016.7 - 2016.8 (2 mo)}
      \resumeItemListStart
        \resumeItem{Facebook粉丝专页 数据挖掘: }\\
        \vspace{1mm}
          {- \href{https://www.sinica.edu.tw/ch}{中央研究院} 是台湾的最高学术机构}\\
          {- 数据挖掘:一个包括台湾\underline{~20~万个Facebook粉丝专页的数据集,包括有~2800~万个贴文内容,以及点赞记录}}\\
          {~~~~- 发现贴文的潜在规律,以及粉丝专页真正的属性}\\
          {~~~~- 技能点包括:\emph{统计学}, \emph{机器学习}, \emph{文本挖掘}, \emph{时间序列分析}}
        \resumeItem{台湾资料科学年会~志愿者} {\color{blue}\href{http://www.datasci.tw}{[网站]}} \\
      \resumeItemListEnd

    \resumeSubheading
      {清华大学电子工程系~科学技术协会
      ~\color{blue}\href{http://www.eesast.com}{[网站]}
      \color{blue}\href{https://github.com/eesast}{[github页面]}
      }
      {中国~北京}
      {\textbf{网站组组长}, 清华大学队式程序设计大赛}{2015.7 - 2016.5 (9 mo)}
      \resumeItemListStart
        \resumeItem{网站开发: \color{blue}\href{http://github.com/eesast/ts17web}{\textbf{[github仓库]}}}\\
        \vspace{1mm}
          {- 带领~2016~清华大学队式程序设计大赛~网站组}\\
          {~~~~- 使用~\href{http://www.yiiframework.com}{\emph{Yii framework}~编写后端 (PHP, MVC)}.}\\
          {~~~~- 使用~\emph{MySQL}, \emph{phpMyAdmin}设计并实现了大部分的数据结构}\\
          {~~~~- 编写大部分的功能:包括 注册、登陆、在线对战、论坛、代码提交, 等等}\\
          {- 协同工作, 使用~\emph{Git}~进行版本控制. }
      \resumeItemListEnd

    \resumeSubheading
      {清华创客空间~\color{blue}\href{http://www.thumaker.cn}{[网站]}}{中国~北京}
      {\textbf{副会长}, 2 年创客领域经验}{2015.9 - Present (1.5 yrs)}
      \resumeItemListStart
        \resumeItem{每周举办~创意周末以及创客比赛: }\\
        \vspace{1mm}
          {- 清华创客空间是中国最大最有影响力的创客空间之一} \\
          {~~~~- 我们曾经收到李克强总理的回信 \color{blue}\href{http://www.moe.edu.cn/publicfiles/business/htmlfiles/moe/moe_838/201505/186765.html}{~\textbf{[媒体报导]}}}\\
          {- 带领~\underline{\emph{项目部}}, \underline{\emph{外联部}}, \underline{\emph{常务部}} 工作 }\\
          {~~~~- 负责每周举办的创意周末以及创客比赛}\\
          {~~~~- 比如 \emph{Arduino}, \emph{Processing}, \emph{3D 建模}, \emph{设计思维}, \emph{头脑风暴}的工作坊、教学, 等等} \\
          {- 带领过一些项目,也是很多创客马拉松比赛的得奖者}\\

      \resumeItemListEnd
      
  \resumeSubHeadingListEnd

% \newpage
  





%--------Competition Projects-----------
\sectionwithbox{部分比赛中的~项目~与~奖项}

  \resumeSubHeadingListStart
    \resumeSubheadingProject{PennyRaiser: 智能募捐箱
        ~\color{blue}\href{http://www.todayfocus.cn/p/4149.html}{\textbf{[媒体报导]}}
        \color{blue}\href{http://bjrb.bjd.com.cn/html/2015-12/07/content_335081.htm}{\textbf{[媒体报导]}}
        \color{black} }
        {2015.4 - 2015.9 (3 day; 6 mo)}
        \resumeItemListStart
          \resumeItem{\emph{2015 清华创客挑战赛}: }{\textbf{冠军}}
          \resumeItem{\emph{2015 中美青年创客大赛}: }{\textbf{三等奖, 世界各地创客团队~Top 5/300} } \\
          \vspace{1mm}
            {- 中美青年创客大赛~是中国、美国最大的创客马拉松比赛
            ~\color{blue}\href{https://www.chinaus-maker.org}{\textbf{[网站]}}
            }\\
            {- 职务: \underline{\textbf{\emph{项目负责人、队长}}} \& \underline{\textbf{\emph{软件工程师}}}} \\
            {~~~- 带领、整合团队, 发起项目} \\
            {~~~- iOS app开发,使用~\emph{Swift}~以及~\emph{Xcode}~} \\
              {~~~~~~- 实现了 \emph{TableView present}, \emph{SideMenu}, \emph{Customized Tab Button}, \emph{Customized Segue Effect}.} \\
              {~~~~~~- 实现二维码支持,让软、硬件进行通信} \\
            {~~~- \emph{设计}, \emph{3D 建模}, \emph{3D 打印}, \emph{激光切割} 打造全部的外观}
        \resumeItemListEnd


    \resumeSubheadingProject{ITE: 可编程手机底座
        ~\color{blue}\href{https://mp.weixin.qq.com/s?__biz=MzA5MzMzODQzMQ==\&mid=2650612087\&idx=1\&sn=12fdb18a73edf403c69fd807c391184a\&mpshare=1\&scene=1\&srcid=0117G3nq1r4Pv0uqSLEmG3oy}{\textbf{[项目介绍]}}
        \color{black} }
        {2016.5 - 2016.5 (3 day)}
        \resumeItemListStart
          \resumeItem{\emph{2016 清华创客挑战赛}: }{\textbf{冠军}} \\
          \vspace{1mm}
            {- 清华创客挑战赛~是清华最大的创客马拉松比赛}\\
            {- 职务: \underline{\textbf{\emph{项目负责人、队长}}} \& \underline{\textbf{\emph{软件工程师}}}} \\
            {~~~- 带领、整合团队, 发起项目} \\
            {~~~- iOS app开发,使用~\emph{Swift}~以及~\emph{Xcode}~} \\
              {~~~~~~- 实现了 \emph{TableView present}, \emph{Customized Tab Button}, \emph{Customized Segue Effect}.} \\
              {~~~~~~- 使用~\emph{CocoaPods}~将开源项目整合进~app~开发} \\
        \resumeItemListEnd

    \resumeSubheadingProject{2015 清华大学队式程序设计大赛 参赛~AI~编写}
        {2015.4 - 2015.5 (1 mo)}
        \resumeItemListStart
          \resumeItem{\emph{2015 清华大学队式程序设计大赛}: }{\textbf{三等奖}} \\
          \vspace{1mm}
            {- 清华大学队式程序设计大赛~是清华大学最大的程序设计比赛之一} \\
            {- 职务: \underline{\textbf{\emph{队长}}}} \\
            {~~~- 带领团队, 思考策略, 使用C++实现, 编写AI运行一个MOBA类的游戏}
        \resumeItemListEnd

  \resumeSubHeadingListEnd










%--------SELECTED HONORS------------
\sectionwithbox{部分荣誉}
  \resumeSubHeadingListStart
    \resumeSubheadinghonor
      {清华创客代表}{, 清华大学 \color{blue}\href{http://lzhbrian.me/posts_figure/maker.jpeg}{~\textbf{[照片]}}}
      {因在创客领域所获得的成就以及贡献}
    \resumeSubheadinghonor
      {启创班成员}{, 清华大学}
      {由清华大学校团委所主持培养创新、创业人才计划,\underline{全校仅录取~30~余人}}
     \resumeSubheadinghonor
      {教育部港澳台学生~一等奖学金}{, 中华人民共和国~教育部}
      {\underline{\textbf{Top 3/50}}, 颁发给在大陆学习的优秀台湾学生}
     \resumeSubheadinghonortwo
      {清华之友~-~郑格如奖学金}{, 清华大学}
      {\underline{\textbf{Top 35/260}}, 获得~\textbf{多个单项奖学金}~后所参评的\textbf{综合奖学金}}
      {包括: \underline{\small\textbf{科技创新优秀奖学金}}, \underline{\small\textbf{社工优秀奖学金}}, \underline{\small\textbf{文艺优秀奖学金}}}
    
  \resumeSubHeadingListEnd


% %--------SELECTED COMPETITIONS------------
% \sectionwithbox{Selected Competitions Awards}
%   \resumeSubHeadingListStart
%     \resumeSubheadinghonor
%       {Tsinghua Maker Challenge 2015,2016}{ : Consecutive Championships 
        
%   \color{blue}\href{http://bjrb.bjd.com.cn/html/2015-12/07/content_335081.htm}{\textbf{[2015 press]}}
%   \color{blue}\href{https://mp.weixin.qq.com/s?__biz=MzA5MzMzODQzMQ==\&mid=2650612087\&idx=1\&sn=12fdb18a73edf403c69fd807c391184a\&mpshare=1\&scene=1\&srcid=0117G3nq1r4Pv0uqSLEmG3oy}{\textbf{[2016 description]}} 
  
%   }
%       {\underline{Championships of 2015, 2016 respectively}. Tsinghua Maker Challenge is the biggest Make-a-thon held in Tsinghua.}
    
    
%      \resumeSubheadinghonortwo
%       {China-US Maker Competitions Final}{ : 3rd Prize 
%       \color{blue}\href{https://www.chinaus-maker.org}{\textbf{[link]}}
%       \color{blue}\href{http://bjrb.bjd.com.cn/html/2015-12/07/content_335081.htm}{\textbf{[press]}}
%       \color{blue}\href{http://www.todayfocus.cn/p/4149.html}{\textbf{[press]}}
%       }
%       {China-US Maker Competition is the \underline{biggest Make-a-thon in China \& US}}
%       {I led my team and won the \underline{5th place out of about 300 teams} all over the world in 2015.}

%      \resumeSubheadinghonor
%       {Tsinghua Teamstyle Program Contest}{ : 3rd Prize}
%       {A programming competition writing AI for games. I led my team and won the 5th place in 2015.}
    
%   \resumeSubHeadingListEnd



%--------PUBLICATIONS------------
\sectionwithbox{论文}
  \resumeSubHeadingListStart
	\resumeSubheadingPublication{H.Yan, \textbf{T.Lin}, G.Wang, Y.Li, H.Zheng, D.Jin, and B.Zhao, On Migratory Behavior in Video Consumption, \emph{Conference Paper Submitted}.}
	\resumeSubheadingPublication{H.Yan, \textbf{T.Lin}, C.Gao, Y.Li, and D.Jin, On the Understanding of User Behaviors Over Multiple Video Content Providers, \emph{Journal Paper Submitted}.}
	\resumeSubheadingPublication{H.Yan, Z.Wang, \textbf{T.Lin}, Y.Li, Y.Wang, and D.Jin, Profiling Users by Online Shopping Behaviors, \emph{Conference Poster Paper Submitted}.}
  \resumeSubHeadingListEnd
  
  
%-------------------------------------------
\end{CJK*}
\end{document}
